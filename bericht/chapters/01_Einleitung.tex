\section{Einleitung}
Die Gefahren für das Internet der Dinge sind vielfältig und Sicherheitskonzepte noch nicht genügend entwickelt oder werden nicht umgesetzt, was eine große Gefahr, für das IoT sowie auch andere Teilnehmer im Internet darstellt.
Die Sicherheit wird, ähnlich wie es bei persönlichen Computern am Anfang geschah, oftmals vernachlässigt. So sind Geräte aus dem IoT aufgrund schlechter Sicherheitsvorkehrungen direkt durch Hackerangriffe gefährdet, die auch persönliche, an das Internet angeschlossene Geräte indirekt treffen können.

\section{Das Internet der Dinge}
Der Begriff \grqq Internet der Dinge\grqq (\grqq Internet of Things\grqq bzw. IoT) beschreibt das Phänomen, dass eine Menge an 'intelligenten' Geräten an das Internet angebunden, aber keine persönlichen Computer sind. Diese Menge umfasst eine Menge an Geräten, die von Produktionsmaschinen von Herstellern über Smart-Home-Systemen bis zu sogenannten Wearables reicht.

\subsection{Chancen}
Das IoT bietet eine Vielzahl an Chancen für den privaten Nutzenden in der moderne bzw. digitalen Welt, so ist eine Verknüpfung autonomer Autos ein Ausblick, der mehr Sicherheit auf den Straßen gewährleisten könnte und die Smart-Home-Systeme bieten eine Vielzahl an Erleichterungen und auch mehr Sicherheit für das tägliche Leben.
\begin{figure}[]
  \includegraphics[width=0.75\columnwidth]{libelium_infographic.png}
  \caption{Chancen für eine smarte Stadt}
\end{figure}
Aber auch für die gewerbliche Nutzung bringt das IoT viele Vorteile mit sich, eine Vereinfachung der Automatisierung wird durch eine starke Vernetzung der einzelnen Maschinen untereinander und mit Planungscomputern ermöglicht.

\section{Sicherheit im Internet der Dinge}
Aufgrund der Tatsache, dass die Technik noch relativ neu ist und erst 2008 mit den Smart-Watches in das Interessenfeld von privaten Usern und somit einer breiteren Masse gerückt ist, ist, laut dem Sicherheitsexperten Alonso, die Sicherheitstechnik in diesem Bereich nur marginal vorhanden \cite{chemaalonso16}. Das wird von Experten der Internet-Sicherheits-Firma Symantec bestätigt: 2014 besaßen viele Geräte laut ihnen keine Ansteuerungsmöglichkeit über einen Dienst wie SSL oder eine beidseitige Authentifizierung der Geräte\cite{insecurity15}. Auch wird von Symantec eine schwache Update-Politik der Hersteller beschrieben, was zu einer Verzögerung beim Schließen von Sicherheitslücken führt. Auch gibt es wenige Sicherheitskonzepte und fast keine Umsetzung.
Laut dem Wirtschaftsinformatiker und Soziologen Jan Peter Kleinhans gibt es für diese fehlende Umsetzung drei Gründe \cite{kleinhans16}. So gibt es wenig ökonomische Anreize: es lohnt sich nicht, ein Sicherheitskonzept umzusetzen, da die Produkte auch ohne diese einen Absatzmarkt finden.
Hersteller, die vorher noch nicht in dem Bereich der Computertechnik gearbeitet haben und dementsprechend kaum Wissen über mögliche Sicherheitskonzepte haben und nicht in diesen Bereichen investieren wollen, sind der zweite Grund. Da hier auch nicht die Bereitschaft zur Auseinandersetzung besteht, wurde z.B. die Verbreitung des Virus Mirai stark vereinfacht, da dieser sich über die Methode \grqq brute forcing\grqq   Zugang verschaffen konnte.
Als dritten Grund nennt Kleinhans die fehlende Wahrnehmung von Bedrohungen und die Unwissenheit über das IoT, wodurch Nutzende nicht wissen, ob ihre Geräte gefährdet oder befallen worden sind.

\section{Gefährdungen im Internet der Dinge}
Die Gefahr geht von Hacker*innen aus, die aus drei Gründen Zugang zu einem Gerät, welches im IoT ist, haben wollen.
Einmal kann ihnen ein solches Gerät Zugang zu Daten verschaffen, so zum Beispiel Aufzeichnungen von Sicherheitskameras oder Plänen, die auf dem Gerät gespeichert sind. Als zweiter Grund lässt sich die Kontrolle über die Rechenleistung nennen. Hier wird das Gerät als Bot in einem Botnetz von einem Betreibenden genutzt, um mit der Rechenkraft entweder einen Dienst über z.B. eine DDoS-Attacke anzugreifen (z.B. Angriff auf Facebook, Netflix und weitere Dienste). Eine andere Möglichkeit wäre die Nutzung der Rechenleistung um rechenintensive Programme auszuführen.
Der dritte Grund ist die Kontrolle über die Geräte, um auf industrieller Ebene Sabotage zu betreiben, indem bestimmte Produktionsmaschinen von ihrem eigentlichen Ablauf abweichen und so Schäden in Millionenhöhe verursachen können.

\section{Fazit}
Die Sicherheitskonzepte sind nicht ausreichend oder gar nicht vorhanden, eine kritische Masse an bewussten Konsumierenden existiert nicht und deshalb sollte eine Legislative ökonomische Anreize schaffen, die Sicherheitskonzepte auszuarbeiten und bestimmte Voraussetzungen setzen, die ein Gerät erfüllen muss, um in dem jeweiligen Staat verkauft werden zu dürfen. 